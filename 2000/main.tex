\documentclass[12pt,-letter paper]{article}
\usepackage{siunitx}
\usepackage{setspace}
\usepackage{gensymb}
\usepackage{xcolor}
\usepackage{caption}
%\usepackage{subcaption}
\doublespacing
\singlespacing
\usepackage[none]{hyphenat}
\usepackage{amssymb}
\usepackage{relsize}
\usepackage[cmex10]{amsmath}
\usepackage{mathtools}
\usepackage{amsmath}
\usepackage{commath}
\usepackage{amsthm}
\interdisplaylinepenalty=2500
%\savesymbol{iint}
\usepackage{txfonts}
%\restoresymbol{TXF}{iint}
\usepackage{wasysym}
\usepackage{amsthm}
\usepackage{mathrsfs}
\usepackage{txfonts}
\let\vec\mathbf{}
\usepackage{stfloats}
\usepackage{float}
\usepackage{cite}
\usepackage{cases}
\usepackage{subfig}
%\usepackage{xtab}
\usepackage{longtable}
\usepackage{multirow}
%\usepackage{algorithm}
\usepackage{amssymb}
%\usepackage{algpseudocode}
\usepackage{enumitem}
\usepackage{mathtools}
%\usepackage{eenrc}
%\usepackage[framemethod=tikz]{mdframed}
\usepackage{listings}
%\usepackage{listings}
\usepackage[latin1]{inputenc}
%%\usepackage{color}{   
%%\usepackage{lscape}
\usepackage{textcomp}
\usepackage{titling}
\usepackage{hyperref}
%\usepackage{fulbigskip}   
\usepackage{tikz}
\usepackage{graphicx}
\lstset{
frame=single,
breaklines=true
}
\let\vec\mathbf{}
\usepackage{enumitem}
\usepackage{graphicx}
\usepackage{siunitx}
\let\vec\mathbf{}
\usepackage{enumitem}
\usepackage{graphicx}
\usepackage{enumitem}
\usepackage{tfrupee}
\usepackage{amsmath}
\usepackage{amssymb}
\usepackage{mwe} % for blindtext and example-image-a in example
\usepackage{wrapfig}
\graphicspath{{figs/}}
\providecommand{\mydet}[1]{\ensuremath{\begin{vmatrix}#1\end{vmatrix}}}
\providecommand{\myvec}[1]{\ensuremath{\begin{bmatrix}#1\end{bmatrix}}}
\providecommand{\cbrak}[1]{\ensuremath{\left\{#1\right\}}}
\providecommand{\brak}[1]{\ensuremath{\left(#1\right)}}
\begin{document}
\begin{enumerate}
\item $AB$ is tangent to the circles $CAMN$ and $NMBD$. $M$ lies between $C$ and $D$ on the line $CD$, and $CD$ is parallel to $AB$. The chords $NA$ and $CM$ meet at $P$; the chords $NB$ and $MD$ meet at $Q$. The rays $CA$ and $DB$ meet at $E$. Prove that $PE = QE$.
\item  $A$,$ B$, $C$ are positive reals with product $1$. Prove that $\brak{A-1+\frac{1}{B}}\brak{B-1+\frac{1}{C}}\brak{C-1+\frac{1}{A}} \leq 1$.	
\item $k$ is a positive real. $N$ is an integer greater than $1$. $N$ points are placed on a line, not all coincident. $A$ $move$ is carried out as follows. Pick any two points $A$ and $B$ which are not coincident. Suppose that $A$ lies to the right of $B$. Replace $B$ by another point $B'$ to the right of $A$ such that $AB' = kBA$. For what values of $k$ can we move the points arbitrarily far to the right by repeated moves?	
\item $100$ cards are numbered $1$ to $100$ $\brak{each card different}$ and placed in $3$ boxes $\brak{at least one card in each box}$. How many ways can this be done so that if two boxes are selected and a card is taken from each, then the knowledge of their sum alone is always sufficient to identify the third box?	
\item Can we find $N$ divisible by just $2000$ different primes, so that $N$ divides $2^N + 1$? [$N$ may be divisible by a prime power.]
\item ${A_{1}} {A_{2}} {A_{3}}$ is an acute-angled triangle. The foot of the altitude from ${A_{i}}$ is ${K_{i}}$ and the incircle touches the side opposite ${A_{i}}$ at ${L_{i}}$. The line ${K_{1}}{K_{2}}$ is reflected in the line ${L_{1}}{L_{2}}$. Similarly, the line ${K_{2}}{K_{3}}$ is reflected in ${L_{2}}{L_{3}}$ and ${K_{3}}{K_{1}}$ is reflected in ${L_{3}}{L_{1}}$. Show that the three new lines form a triangle with vertices on the incircle.	
\end{enumerate}
\end{document}
