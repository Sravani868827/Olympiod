\documentclass[12pt,-letter paper]{article}
\usepackage{siunitx}
\usepackage{setspace}
\usepackage{gensymb}
\usepackage{xcolor}
\usepackage{caption}
%\usepackage{subcaption}
\doublespacing
\singlespacing
\usepackage[none]{hyphenat}
\usepackage{amssymb}
\usepackage{relsize}
\usepackage[cmex10]{amsmath}
\usepackage{mathtools}
\usepackage{amsmath}
\usepackage{commath}
\usepackage{amsthm}
\interdisplaylinepenalty=2500
%\savesymbol{iint}
\usepackage{txfonts}
%\restoresymbol{TXF}{iint}
\usepackage{wasysym}
\usepackage{amsthm}
\usepackage{mathrsfs}
\usepackage{txfonts}
\let\vec\mathbf{}
\usepackage{stfloats}
\usepackage{float}
\usepackage{cite}
\usepackage{cases}
\usepackage{subfig}
%\usepackage{xtab}
\usepackage{longtable}
\usepackage{multirow}
%\usepackage{algorithm}
\usepackage{amssymb}
%\usepackage{algpseudocode}
\usepackage{enumitem}
\usepackage{mathtools}
%\usepackage{eenrc}
%\usepackage[framemethod=tikz]{mdframed}
\usepackage{listings}
%\usepackage{listings}
\usepackage[latin1]{inputenc}
%%\usepackage{color}{   
%%\usepackage{lscape}
\usepackage{textcomp}
\usepackage{titling}
\usepackage{hyperref}
%\usepackage{fulbigskip}   
\usepackage{tikz}
\usepackage{graphicx}
\lstset{
  frame=single,
  breaklines=true
}
\let\vec\mathbf{}
\usepackage{enumitem}
\usepackage{graphicx}
\usepackage{siunitx}
\let\vec\mathbf{}
\usepackage{enumitem}
\usepackage{graphicx}
\usepackage{enumitem}
\usepackage{tfrupee}
\usepackage{amsmath}
\usepackage{amssymb}
\usepackage{mwe} % for blindtext and example-image-a in example
\usepackage{wrapfig}
\graphicspath{{figs/}}
\providecommand{\mydet}[1]{\ensuremath{\begin{vmatrix}#1\end{vmatrix}}}
\providecommand{\myvec}[1]{\ensuremath{\begin{bmatrix}#1\end{bmatrix}}}
\providecommand{\cbrak}[1]{\ensuremath{\left\{#1\right\}}}
\providecommand{\brak}[1]{\ensuremath{\left(#1\right)}}
\begin{document}
\begin{enumerate}
\item In the convex quadrilateral $ABCD$, the diagonals $AC$ and $BD$ are perpendicular and the opposite sides $AB$ and $DC$ are not parallel. Suppose that the point $P$, where the perpendicular bisectors of $AB$ and $DC$ meet, is inside $ABCD$. Prove that $ABCD$ is a cyclic quadrilateral if and only if the triangles $ABP$ and $CDP$ have equal areas.
\item In a competition, there are $a$ contestants and $b$ judges, where $ b \geq 3$ is an odd integer. Each judge rates each contestant as either "pass" or "fail". Suppose $k$ is a number such that, for any two judges, their ratings coincide for at most $k$ contestants. Prove that $ \frac{k}{a} \geq \frac{\brak{b-1}}{\brak{2b}}$.
\item For any positive integer $n$, let d{\brak{n}} denote the number of positive divisors of $n$ (including $1$ and $n$ itself). Determine all positive integers k such that $ \frac{d\brak{n^2}} {d\brak{n}}  = k$ for some $n$.
\item Determine all pairs \brak{a, b} of positive integers such that $ab^2 + b + 7$ divides $a^2b + a + b$.
\item Let $I$ be the incenter of triangle $ABC$. Let the incircle of $ABC$ touch the sides $BC$, $CA$, and $AB$ at $K$, $L$, and $M$, respectively. The line through $B$ parallel to $MK$ meets the lines $LM$ and $LK$ at $R$ and $S$, respectively. Prove that angle $RIS$ is acute.
\item  Consider all functions $f$ from the set $N$ of all positive integers into itself satisfying $f\brak{t^2f\brak{s}} = s\brak{f\brak{t}}^2$ for all $s$ and $t$ in $N$. Determine the least possible value of ${f\brak{1998}}$.	
\end{enumerate}
\end{document}		
