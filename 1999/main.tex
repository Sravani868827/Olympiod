\documentclass[12pt,-letter paper]{article}
\usepackage{siunitx}
\usepackage{setspace}
\usepackage{gensymb}
\usepackage{xcolor}
\usepackage{caption}
%\usepackage{subcaption}
\doublespacing
\singlespacing
\usepackage[none]{hyphenat}
\usepackage{amssymb}
\usepackage{relsize}
\usepackage[cmex10]{amsmath}
\usepackage{mathtools}
\usepackage{amsmath}
\usepackage{commath}
\usepackage{amsthm}
\interdisplaylinepenalty=2500
%\savesymbol{iint}
\usepackage{txfonts}
%\restoresymbol{TXF}{iint}
\usepackage{wasysym}
\usepackage{amsthm}
\usepackage{mathrsfs}
\usepackage{txfonts}
\let\vec\mathbf{}
\usepackage{stfloats}
\usepackage{float}
\usepackage{cite}
\usepackage{cases}
\usepackage{subfig}
%\usepackage{xtab}
\usepackage{longtable}
\usepackage{multirow}
%\usepackage{algorithm}
\usepackage{amssymb}
%\usepackage{algpseudocode}
\usepackage{enumitem}
\usepackage{mathtools}
%\usepackage{eenrc}
%\usepackage[framemethod=tikz]{mdframed}
\usepackage{listings}
%\usepackage{listings}
\usepackage[latin1]{inputenc}
%%\usepackage{color}{   
%%\usepackage{lscape}
\usepackage{textcomp}
\usepackage{titling}
\usepackage{hyperref}
%\usepackage{fulbigskip}   
\usepackage{tikz}
\usepackage{graphicx}
\lstset{
frame=single,
breaklines=true
}
\let\vec\mathbf{}
\usepackage{enumitem}
\usepackage{graphicx}
\usepackage{siunitx}
\let\vec\mathbf{}
\usepackage{enumitem}
\usepackage{graphicx}
\usepackage{enumitem}
\usepackage{tfrupee}
\usepackage{amsmath}
\usepackage{amssymb}
\usepackage{mwe} % for blindtext and example-image-a in example
\usepackage{wrapfig}
\graphicspath{{figs/}}
\providecommand{\mydet}[1]{\ensuremath{\begin{vmatrix}#1\end{vmatrix}}}
\providecommand{\myvec}[1]{\ensuremath{\begin{bmatrix}#1\end{bmatrix}}}
\providecommand{\cbrak}[1]{\ensuremath{\left\{#1\right\}}}
\providecommand{\brak}[1]{\ensuremath{\left(#1\right)}}
\begin{document}
\begin{enumerate}
\item Determine all finite sets $S$ of at least three points in the plane which satisfy the following condition:\\for any two distinct points $A$ and $B$ in $S$, the perpendicular bisector of the line segment $AB$ is an axis of symmetry for $S$.
\item  Let $n$ be a fixed integer, with $n \geq 2$.\\ $\brak{a}$ Determine the least constant $C$ such that the inequality \\\begin{align*}\sum_{1\leq i<j\leq n} {x_{i}} {x_{j}} \brak{x_{i}^2 + x_{j}^2} \leq C \brak{\sum_{1\leq i\leq n} x_{i}}^4\end{align*}\\ holds for all real numbers ${x_{1}},....,{x_{n}} \geq 0$.\\ $\brak{b}$ For this constant $C$, determine when equality holds.
\item Consider an $n \times n$ square board, where $n$ is a fixed even positive integer. The board is divided into $n^2$ unit squares. We say that two different squares on the board are adjacent if they have a common side.\\$N$ unit squares on the board are marked in such a way that every square \brak{marked or unmarked} on the board is adjacent to at least one marked square.\\Determine the smallest possible value of $N$.
\item Determine all pairs \brak{n,p} of positive integers such that\\$p$ is a prime,\\$n$ not exceeded $2p$,and\\ $\brak{p-1}^n+1$ is divisible by $n^{p-1}$.
\item Two circles ${G_{1}}$ and ${G_{2}}$ are contained inside the circle $G$, and are tangent to $G$ at the distinct points $M$ and $N$, respectively. ${G_{1}}$ passes through the center of ${G_{2}}$. The line passing through the two points of intersection of ${G_{1}}$ and ${G_{2}}$ meets $G$ at $A$ and $B$. The lines $MA$ and $MB$ meet ${G_{1}}$ at $C$ and $D$, respectively.\\ Prove that $CD$ is tangent to ${G_{2}}$.	
\item Determine all functions $f: \textbf{R} \to \textbf{R}$ such that\\$f\brak{x - f\brak{y}} = f\brak{f\brak{y}} + xf\brak{y} + f\brak{x} - 1$
\end{enumerate}
\end{document}
